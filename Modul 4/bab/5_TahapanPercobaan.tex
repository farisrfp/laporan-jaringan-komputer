% Bagian Tahapan Percobaan
\section*{Tahapan Percobaan} % Jika ada tahapan percobaan

    \subsection*{Konfigurasi PC 1}

    \begin{enumerate}
        \item Menghubungkan PC ke Router ke Winbox dengan login sebagai "admin".
        \item Mengkonfigurasi Router sebagai DHCP Client dengan memilih interface yang terhubung ke Internet dan melakukan tes ping ke IP 8.8.8.8.
        \item Menambahkan IP Address pada Router 1 dengan memilih interface yang terhubung ke PC (ether4).
        \item Mengatur IP pada PC 1 dengan mengubah pengaturan ethernet menjadi manual dan memastikan IP PC 1 berada pada satu jaringan dengan IP lokal yang diinginkan.
        \item Melakukan Tes Ping antara Router dan PC1 untuk memastikan keduanya terkoneksi.
        \item Membuat PPTP Server pada Router menggunakan Default Profile “default-encryption”.
        \item Mengkonfigurasi PPTP Client pada Router dengan menambahkan Secret dan mengisi Nama, Password, Service, Profile, Local Address, dan Remote Address.
        \item Menambahkan Routing Statis untuk menghubungkan PC1 dengan Internet melalui alamat Gateway pada router 1.
        \item Mengaktifkan Layanan NAT untuk menghubungkan jaringan lokal ke jaringan publik dengan memilih Out Interface (ether6) dan mengatur Action ke “masquerade”.
        \item Melakukan Tes Ping untuk menguji koneksi antar Router dan PC, memastikan kedua Router dan PC sudah terhubung.
    \end{enumerate}

    \subsection*{Konfigurasi PC 2}

    \begin{enumerate}
        \item Menghubungkan PC2 ke Internet menggunakan wifi ITS.
        \item Memeriksa IP yang diterima PC2 dari ITS melalui pengaturan jaringan.
        \item Masuk ke Pengaturan VPN di PC2 dan membuat koneksi VPN baru antara PC2 dengan Host Server menggunakan VPN type PPTP.
        \item Menghubungkan ke VPN dengan klik tombol Connect.
    \end{enumerate}

    \subsection*{Pengujian Konfigurasi}

    \begin{enumerate}
        \item Melakukan Tes Ping untuk Memastikan Koneksi VPN PPTP dengan ping ke alamat PC1.
    \end{enumerate}