\section*{Tugas Modul} % Jika ada Tugas Modul
\begin{enumerate}
    \item Apa itu PPTP dan bagaimana cara kerjanya?
    
    PPTP (Point-to-Point Tunneling Protocol) adalah protokol VPN yang telah lama digunakan karena kemudahan konfigurasinya dan kompatibilitasnya dengan berbagai sistem operasi dan perangkat. PPTP bekerja dengan melibatkan pembuatan tunnel antara klien dan server, autentikasi pengguna, dan enkripsi data menggunakan MPPE. Meskipun mudah diimplementasikan, PPTP memiliki kelemahan keamanan dibandingkan dengan protokol VPN yang lebih modern seperti L2TP/IPsec atau OpenVPN. Di MikroTik, konfigurasi PPTP melibatkan pengaturan server dan klien VPN, serta penyesuaian routing dan aturan firewall untuk mendukung lalu lintas VPN. Oleh karena itu, meskipun masih digunakan, PPTP kurang direkomendasikan untuk kebutuhan keamanan yang lebih tinggi.

    \item Apa kelebihan dan kekurangan dari penggunaan PPTP dibandingkan protokol VPN lainnya seperti L2TP atau OpenVPN?
    
    PPTP unggul dalam hal kemudahan konfigurasi dan kecepatan, tetapi memiliki kelemahan keamanan yang signifikan. L2TP/IPsec menawarkan keamanan yang lebih baik dan kemampuan melewati firewall yang lebih baik tetapi lebih sulit dikonfigurasi dan sedikit lebih lambat. OpenVPN, di sisi lain, memberikan keamanan terbaik dan fleksibilitas tinggi, meskipun konfigurasi awalnya lebih kompleks dan memerlukan perangkat lunak tambahan untuk digunakan pada banyak perangkat. Pilihan protokol tergantung pada kebutuhan spesifik, apakah prioritasnya adalah kemudahan penggunaan, kecepatan, atau tingkat keamanan yang tinggi.
\end{enumerate}