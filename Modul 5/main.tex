\documentclass{article}
\usepackage[margin=1in]{geometry}
\usepackage[utf8]{inputenc}
\usepackage{graphicx} 
\usepackage{fancyhdr}
\usepackage{enumitem}
\usepackage{lipsum}
\usepackage[bahasa]{babel}
\usepackage{subcaption}
\usepackage{float}
\usepackage{indentfirst}
\usepackage{subcaption}
\usepackage{tcolorbox}

\title{Laporan Praktikum Jaringan Komputer \\ IPv6} % Ganti sesuai modul praktikum yang diikuti
\author{
  Yohana Magdalena Franklin Harianja - 5024221012\\
  \and
  Faris Rafi Pramana - 5024201004\\
  \and
  Xyz Frizy Firstyaji - 5024221073\\
}
\date{}


\fancypagestyle{firstpageheader}{
  \fancyhf{} 
  \fancyhead[L]{\includegraphics[height=1.5cm]{img/logodepart.png}} 
  \fancyhead[R]{Institut Teknologi Sepuluh Nopember \\ Departemen Teknik Komputer \\ Laboratorium Multimedia \emph{Internet of Things}} 
  \renewcommand{\headrulewidth}{0pt} 
  \fancyfoot[C]{%
    \includegraphics[width=\textwidth]{img/footer.png}
  }
  \renewcommand{\footrulewidth}{0pt} % No line in the footer
}


\fancyhf{} 
\fancyfoot[C]{%
  \includegraphics[width=\textwidth]{img/footer.png}
}
\renewcommand{\headrulewidth}{0pt} 
\renewcommand{\footrulewidth}{0pt} 

\pagestyle{fancy}
\begin{document}
\selectlanguage{bahasa}
\maketitle
\thispagestyle{firstpageheader}

% Bagian Pendahuluan
\section*{Pendahuluan}
\lipsum[1]
% Bagian Analisis Hasil Percobaan
\section*{Tujuan}
\indent
Pada gambar \ref{fig:inirujukan} dapat dilihat bahwa pada tabel \ref{tab:labeltabel} terdapat data yang menunjukkan bahwa \lipsum[1]

\begin{table}[h]
  \centering
  \caption{Caption tabelnya}
  \label{tab:labeltabel}
  \begin{tabular}{|c|c|c|}
    \hline
    Kolom 1 & Kolom 2 & Kolom 3 \\
    \hline
    Data 1  & Data 2  & Data 3  \\
    Data 4  & Data 5  & Data 6  \\
    \hline
  \end{tabular}
\end{table}
% Bagian Alat dan Bahan
\section*{Alat dan Bahan}
\indent
\begin{enumerate}
    \item Alat 1
    \item Alat 2
\end{enumerate}

Jika menggunakan Tabel
\begin{table}[h]
    \centering
    \caption{Caption Tabelnya}
    \label{tab:labelini}
    \begin{tabular}{|c|c|c|}
        \hline
        Kolom 1 & Kolom 2 & Kolom 3 \\
        \hline
        Data 1  & Data 2  & Data 3  \\
        Data 4  & Data 5  & Data 6  \\
        \hline
    \end{tabular}
\end{table}
% Bagian Topologi
\section*{Topologi} % Jika ada lampiran

Routing berikut ini \lipsum[1]

\begin{figure}[H]
    \centering
    \includegraphics[width=0.2\linewidth]{img/contohgambar.png}
    \caption{Ini Caption}
    \label{fig:initopologi}
\end{figure}

% Bagian Tahapan Percobaan
\section*{Tahapan Percobaan} 

\subsection*{Konfigurasi PC 1}

\begin{enumerate}
    \item Menghubungkan Winbox pada PC1 ke Router, dengan login menggunakan \texttt{admin}, serta memilih Router yang ingin disambungkan melalui fitur \textbf{Neighbors}.
    \item Mengaktifkan layanan IPv6 pada Router1 melalui menu \textbf{System} dan \textbf{Packages}, serta melakukan reboot pada Router1.
    \item Menambahkan alamat IPv6 pada Router1 untuk menghubungkan ke PC1 melalui interface yang sesuai, misalnya \textbf{ether4}.
    \item Mengubah pengaturan IPv6 pada PC1 dari otomatis ke manual, memastikan IPv6 PC1 berada dalam jaringan yang sama dengan Router1, serta mengisi Gateway dengan alamat IPv6 dari Router1.
    \item Melakukan uji konektivitas dengan melakukan ping antara Router1 dan PC1 untuk memastikan kedua perangkat sudah saling terhubung.
    \item Mengonfigurasikan IPv6 pada Router1 untuk menghubungkan Router1 dengan Router2, menambahkan alamat IPv6 pada interface yang sesuai, misalnya \textbf{ether2}.
    \item Menghubungkan kedua jaringan menggunakan routing statis melalui menu \textbf{IPv6 > Routes}, dan menambahkan route baru dengan alamat jaringan tujuan melalui Gateway pada Router2.
\end{enumerate}

\subsection*{Konfigurasi PC 2}

\begin{enumerate}
    \item Menghubungkan Winbox pada PC2 ke Router, dengan login menggunakan \texttt{admin}, serta memilih Router yang ingin disambungkan melalui fitur \textbf{Neighbors}.
    \item Mengaktifkan layanan IPv6 pada Router2 melalui menu \textbf{System} dan \textbf{Packages}, serta melakukan reboot pada Router2.
    \item Menambahkan alamat IPv6 pada Router2 untuk menghubungkan ke PC2 melalui interface yang sesuai, misalnya \textbf{ether4}.
    \item Mengubah pengaturan IPv6 pada PC2 dari otomatis ke manual, memastikan IPv6 PC2 berada dalam jaringan yang sama dengan Router2, serta mengisi Gateway dengan alamat IPv6 dari Router2.
    \item Melakukan uji konektivitas dengan melakukan ping antara Router2 dan PC2 untuk memastikan kedua perangkat sudah saling terhubung.
    \item Mengonfigurasikan IPv6 pada Router2 untuk menghubungkan Router2 dengan Router1, menambahkan alamat IPv6 pada interface yang sesuai, misalnya \textbf{ether2}.
    \item Menghubungkan kedua jaringan menggunakan routing statis melalui menu \textbf{IPv6 > Routes}, dan menambahkan route baru dengan alamat jaringan tujuan melalui Gateway pada Router1.
\end{enumerate}
% Bagian Hasil Percobaan
\section*{Hasil Percobaan} % Jika ada hasil percobaan

\subsection*{Routing Statis}
\begin{figure}[H]
  \centering
  \includegraphics[width=0.5\linewidth]{img/contohgambar.png}
  \caption{Ini Caption}
  \label{fig:iniroutingstatis}
\end{figure}

\subsection*{Routing Dinamis}
\begin{figure}[H]
  \centering
  \includegraphics[width=0.5\linewidth]{img/contohgambar.png}
  \caption{Ini Caption}
  \label{fig:iniroutingdinamis}
\end{figure}

% Bagian Tugas Modul
\section*{Tugas Modul} % Jika ada Tugas Modul
\begin{enumerate}
  \item bla bla bla ?
        \begin{itemize}
          \item Jawabannya ya
        \end{itemize}
  \item bla bla bla ?
        \begin{itemize}
          \item Jawabannya ya
        \end{itemize}
\end{enumerate}
% Bagian Kesimpulan
\section*{Kesimpulan}

Berdasarkan \lipsum[1]

% Bagian Lampiran
\section*{Lampiran} % Jika ada lampiran


\end{document}