% Bagian Kesimpulan
\section*{Kesimpulan}

IPv6 adalah standar protokol terbaru untuk mengidentifikasi dan mengarahkan alamat jaringan dalam jaringan komputer. IPv6 hadir sebagai solusi atas keterbatasan jumlah alamat yang tersedia dalam pendahulunya, IPv4. Dengan format alamat 128-bit, IPv6 mampu menyediakan jumlah alamat yang jauh lebih besar, sehingga dapat mengakomodasi pertumbuhan pesat internet dan kebutuhan alamat yang terus meningkat. Selain itu, IPv6 juga menawarkan sejumlah fitur tambahan, seperti peningkatan keamanan, pemantauan aliran lalu lintas yang lebih baik, dan kualitas layanan yang lebih optimal. Fitur-fitur ini menjadikan IPv6 sebagai solusi jangka panjang yang penting untuk memenuhi kebutuhan jaringan di masa depan.