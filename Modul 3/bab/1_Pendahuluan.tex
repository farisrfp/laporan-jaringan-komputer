% Bagian Pendahuluan
\section*{Pendahuluan}

Dalam era digital yang semakin maju, kebutuhan akan jaringan internet yang cepat dan andal semakin meningkat. Berbagai aplikasi seperti video streaming, gaming online, dan transfer data besar menuntut alokasi bandwidth yang efisien agar semua pengguna dapat menikmati layanan internet yang optimal. Namun, tanpa manajemen yang tepat, beberapa aplikasi dapat mendominasi bandwidth, merugikan pengguna lain.

Untuk mengatasi masalah ini, Quality of Service (QoS) hadir sebagai solusi yang efektif. QoS memungkinkan pengaturan prioritas dan alokasi bandwidth yang adil bagi berbagai jenis lalu lintas jaringan. Salah satu metode QoS yang sederhana namun efektif adalah Simple Queue. Dengan Simple Queue, administrator jaringan dapat membagi bandwidth secara proporsional dan memprioritaskan aplikasi penting, sehingga menjamin kualitas layanan yang konsisten bagi semua pengguna.
