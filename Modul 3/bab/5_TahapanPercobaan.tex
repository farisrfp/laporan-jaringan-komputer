% Bagian Tahapan Percobaan
\section*{Tahapan Percobaan} % Jika ada tahapan percobaan

\subsection*{Tahap 1: Konfigurasi Awal}

\begin{enumerate}
    \item \textbf{Buka WinBox:} Jalankan aplikasi WinBox pada PC Anda.
    \item \textbf{Sambungkan ke Router:}
    \begin{itemize}
        \item Isi bagian Login dengan "admin".
        \item Klik "Neighbors", lalu "Refresh".
        \item Pilih router target dan klik "Connect".
    \end{itemize}
    \item \textbf{Set DHCP Client:}
    \begin{itemize}
        \item Pada bagian interface, pilih "ether2" dan klik "Apply", lalu "OK".
        \item Pada window DHCP Client, tambahkan IP Address router (antara PC dan Router) dengan mengisi address dan memilih interface yang terhubung ke "ether2". Klik "Apply" dan "OK".
    \end{itemize}
\end{enumerate}

\subsection*{Tahap 2: Konfigurasi DHCP Server}

\begin{enumerate}
    \item \textbf{Set DHCP Server:} Pilih interface yang akan menjadi server pada DHCP Setup.
    \item \textbf{DHCP Address Space:} Isi alamat network yang diinginkan dengan akhiran "0" pada IP Address.
    \item \textbf{Gateway for DHCP Network:} Isi port router yang akan menghubungkan router dengan network address tujuan.
    \item \textbf{Address to Give Out:} Isi range IP Address yang akan didaftarkan pada perangkat ja-ringan (xxx.xxx.xx.[3-255]). Klik "Next" pada bagian DNS Server.
    \item \textbf{Cek IP Setting:} Pastikan IP setting untuk koneksi ethernet dalam mode Automatic (DHCP).
\end{enumerate}

\subsection*{Tahap 3: Konfigurasi Firewall dan NAT}

\begin{enumerate}
    \item \textbf{Buka Firewall:} Buka tab Firewall.
    \item \textbf{Buat NAT Baru:}
    \begin{itemize}
        \item Tambahkan pada opsi chain dan pilih "srcnat".
        \item Pilih "ether6" pada Out Interface. Klik "Apply".
        \item Tambahkan action dengan set menjadi "masquerade". Klik "Apply" dan "OK".
    \end{itemize}
\end{enumerate}

\subsection*{Tahap 4: Pengujian Koneksi}

\begin{enumerate}
    \item \textbf{Tes PING:} Lakukan tes PING ke 8.8.8.8 untuk memeriksa koneksi ke jaringan publik.
    \item \textbf{Speed Test:} Jika PING berhasil, lakukan speed test pada browser.
\end{enumerate}

\subsection*{Tahap 5: Pengujian Pembatasan Bandwidth}

\begin{enumerate}
    \item \textbf{Buka Queue:} Buka tab queue.
    \item \textbf{Batasi Bandwidth:}
    \begin{itemize}
        \item Pilih IP Address perangkat yang ingin dibatasi.
        \item Batasi target upload. Klik "Apply" dan "OK".
    \end{itemize}
    \item \textbf{Ulangi Speed Test:} Lakukan speed test lagi untuk melihat efek pembatasan bandwidth.
\end{enumerate}