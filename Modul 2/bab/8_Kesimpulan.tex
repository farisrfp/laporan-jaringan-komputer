% Bagian Kesimpulan
\section*{Kesimpulan}

Praktikum ini berhasil menunjukkan perbedaan mendasar antara routing statis dan dinamis dalam jaringan komputer. Routing statis, ideal untuk jaringan kecil dengan perubahan konfigurasi minimal, menawarkan kesederhanaan dan efisiensi sumber daya. Namun, keterbatasannya dalam beradaptasi dengan perubahan topologi jaringan menjadikannya kurang cocok untuk jaringan yang lebih besar dan kompleks.

Di sisi lain, routing dinamis, khususnya protokol RIP, menunjukkan kemampuan adaptasi yang sangat baik terhadap perubahan jaringan. Kemampuannya untuk menemukan jalur terbaik secara otomatis meningkatkan efisiensi dan keandalan jaringan, terutama dalam skala besar. Walaupun konfigurasinya lebih rumit, manfaatnya dalam mengelola jaringan dinamis menjadikannya pilihan yang sangat berharga.

Secara keseluruhan, pemahaman tentang kedua jenis routing ini sangat penting dalam merancang dan mengelola jaringan komputer yang efektif dan efisien. Pemilihan antara routing statis dan dinamis harus mempertimbangkan ukuran, kompleksitas, dan kebutuhan adaptabilitas jaringan yang akan dikelola.