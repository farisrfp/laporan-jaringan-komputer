% Bagian Pendahuluan
\section*{Pendahuluan}

Dalam dunia jaringan komputer yang terus berkembang, pemahaman tentang routing menjadi krusial. Routing adalah proses yang menentukan bagaimana paket data berpindah dari satu jaringan ke jaringan lain hingga mencapai tujuan akhirnya. Konfigurasi routing yang tepat memastikan kelancaran komunikasi antar perangkat dalam jaringan yang berbeda, baik itu jaringan lokal maupun jaringan yang lebih luas seperti internet.

Modul ini akan membahas dua jenis routing yang umum digunakan, yaitu routing statis dan routing dinamis. Routing statis melibatkan konfigurasi manual jalur-jalur yang harus diambil oleh paket data, sementara routing dinamis secara otomatis menyesuaikan jalur berdasarkan kondisi jaringan yang berubah-ubah. Dengan memahami kedua jenis routing ini, kita dapat memilih metode yang paling sesuai dengan kebutuhan dan karakteristik jaringan kita.
