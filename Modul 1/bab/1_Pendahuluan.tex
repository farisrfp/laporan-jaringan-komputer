% Bagian Pendahuluan
\section*{Pendahuluan}

% Module Wireless

Pada modul ini, kita akan mempelajari tentang \textit{wireless}. \textit{Wireless} adalah teknologi yang memungkinkan dua atau lebih perangkat untuk berkomunikasi tanpa menggunakan kabel. Teknologi ini memungkinkan kita untuk berkomunikasi dengan perangkat lain tanpa harus terhubung dengan kabel. Teknologi ini sangat berguna dalam kehidupan sehari-hari, seperti pada saat kita menggunakan \textit{smartphone}

Teknologi \textit{wireless} telah merevolusi cara perangkat berkomunikasi tanpa menggunakan kabel, membuka peluang untuk berbagai aplikasi, seperti penggunaan smartphone dan perangkat IoT. Dalam jaringan komputer, ada tiga jenis utama koneksi \textit{wireless}: Point-to-Point Protocol (PPP), Point-to-Multipoint, dan Wireless Bridging. PPP adalah protokol yang menghubungkan dua node jaringan secara langsung, sering digunakan untuk menghubungkan jaringan di dua gedung atau Base Transceiver Station (BTS). Protokol ini mendukung fitur-fitur penting seperti otentikasi, enkripsi, dan kompresi, memastikan koneksi yang aman dan stabil. Sementara itu, Point-to-Multipoint memungkinkan satu sumber terhubung dengan banyak node, yang umum digunakan untuk menciptakan jaringan WiFi atau hotspot di area yang membutuhkan akses \textit{wireless} bagi banyak pengguna, seperti di rumah, kantor, atau tempat umum.

Wireless Bridging adalah jenis koneksi \textit{wireless} yang menghubungkan dua segmen LAN, membuatnya berada dalam subnet yang sama tanpa perlu kabel fisik. Pendekatan ini sering digunakan untuk menghubungkan dua jaringan yang terpisah secara fisik, seperti dua gedung yang berdekatan. Setiap jenis koneksi ini memiliki kelebihan dan aplikasi spesifik, sehingga pemilihan di antara mereka harus mempertimbangkan kebutuhan jaringan dan kondisi lingkungan. Dengan memahami perbedaan dan aplikasi terbaik untuk masing-masing jenis koneksi ini, kita dapat merancang jaringan \textit{wireless} yang andal, aman, dan berkinerja tinggi untuk berbagai keperluan.