% Bagian Tahapan Percobaan
\section*{Tahapan Percobaan} % Jika ada tahapan percobaan

Berikut adalah langkah-langkah yang dilakukan untuk menghubungkan PC ke router menggunakan Winbox dan mengatur beberapa konfigurasi pada router.

\subsection*{Langkah 1: Hubungkan PC ke Router}
\begin{enumerate}
    \item \textbf{Hubungkan Kabel LAN}:
    Sambungkan kabel LAN dari port Ethernet PC ke port Ethernet pada router.
\end{enumerate}

\subsection*{Langkah 2: Buka Aplikasi Winbox}
\begin{enumerate}
    \item \textbf{Buka Winbox}:
    Buka aplikasi Winbox di PC Anda.
\end{enumerate}

\subsection*{Langkah 3: Hubungkan Winbox ke Router}
\begin{enumerate}
    \item \textbf{Pilih Router dari Neighbor List}:
    \begin{itemize}
        \item Di Winbox, pilih tab 'Neighbors'.
        \item Klik 'Refresh' untuk memuat daftar router yang terhubung.
        \item Pilih router yang ingin dihubungkan dari daftar yang muncul.
    \end{itemize}
    
    \item \textbf{Hubungkan ke Router}:
    Setelah memilih router, klik 'Connect' untuk terhubung ke router.
\end{enumerate}

\subsection*{Langkah 4: Atur Alamat IP pada Router}
\begin{enumerate}
    \item \textbf{Buka Pengaturan IP Address}:
    \begin{itemize}
        \item Klik menu 'IP'.
        \item Pilih 'Addresses' dari daftar menu.
    \end{itemize}
    
    \item \textbf{Tambah Alamat IP Baru}:
    \begin{itemize}
        \item Pada jendela 'Addresses', klik tanda tambah (+).
        \item Isi 'Address' dengan alamat IP yang ingin digunakan.
        \item Isi 'Network' dengan jaringan yang sesuai.
        \item Pilih 'Interface' yang akan digunakan (misalnya, 'ether1' untuk port Ethernet pertama).
        \item Klik 'Apply', lalu 'OK'.
    \end{itemize}
\end{enumerate}

\subsection*{Langkah 5: Mengaktifkan WLAN pada Router}
\begin{enumerate}
    \item \textbf{Aktifkan Wireless Interface}:
    \begin{itemize}
        \item Klik menu 'Wireless'.
        \item Pada tab 'Interfaces', pilih 'wlan1' atau interface WLAN yang tersedia.
        \item Klik kotak centang di samping interface untuk mengaktifkannya.
        \item Klik 'Apply', lalu 'OK'.
    \end{itemize}
\end{enumerate}

\subsection*{Langkah 6: Atur WLAN Mode dan SSID}
\begin{enumerate}
    \item \textbf{Atur Mode Wireless}:
    \begin{itemize}
        \item Tetap di tab 'Wireless', pilih 'wlan1' (atau interface WLAN lainnya).
        \item Di bagian 'Wireless', pilih mode yang diinginkan. Untuk Point to Point, pilih 'bridge', dan untuk Point to Multi Point, pilih 'ap bridge'.
        \item Ubah 'SSID' sesuai keinginan Anda.
        \item Klik 'Apply', lalu 'OK'.
    \end{itemize}
\end{enumerate}